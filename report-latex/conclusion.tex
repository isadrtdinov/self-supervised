В данной работе мы проводим сравнительный анализ нескольких постановок обучения нейронных сетей: обучения в парадигме self-supervised learning на примере алгоритма SimCLR, обучения с учителем на классификацию изображений и обучения со случайной разметкой по классам. Мы изучаем динамику обучения и сложность объектов с точки зрения выучивания их нейронными сетями. Мы показываем, что в контексте всех постановок обучающие объекты имеют неодинаковую сложность. Наряду с этим, обучению с учителем свойственнен наибольший разброс сложности объектов. При фиксированных аугментациях изображений сложности объектов в методе SimCLR напоминают обучение с учителем, однако усреднение по аугментациям выравнивает сложности, что позволяет соотнести данный метод и обучение со случайной разметкой.

Мы приводим эмпирическое подтверждение того, что две указанные постановки имеют схожую динамику обучения. Мы объясняем данный феномен тем, что нейронные сети и в методе SimCLR, и при обучении со случайной разметкой должны научиться распозавать обучающие объекты. Данное предположение отличает указанные постановки от обучения с учителем, где для правильной классификации нейронной сети достаточно выделить фрагменты изображений, специфичные для каждого класса. Тем не менее, наш анализ обученных векторных представлений показывает, что представления метода SimCLR имеют четкую семантическую структуру, в отличие от представлений обучения со случайной разметкой. Данное наблюдение устанавливает связь между обучением с учителем и методом SimCLR, несмотря на то что при обучении последнего не используется разметка по классам изображений.

В заключение можно сказать, что динамика нейронных сетей, обученных в парадигме self-supervised learning, похожа на динамику обучения со случайной разметкой, однако поскольку нейронные сети обучаются не на случайный шум, а на некоторую содержательную задачу, то предобучение таким способом является полезным для использования нейронных сетей на других задачах компьютерного зрения.
